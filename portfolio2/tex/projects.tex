\section{Projects}
\subsection{Expression Solver}
\info{Expression Solver}{5}{
  Write a program that takes, as input, a fully parenthesized,
  arithmetic expression and converts it to a binary expression tree.
  Your program should display the tree in some way and also print the
  value associated with the root. For an additional challenge, allow
  for the leaves to store variables of the form x1, x2, x3, and so on,
  which are initially 0 and which can be updated interactively by your
  program, with the corresponding update in the printed value of the
  root of the expression tree.  
  Begin with the "For an additional challenge..." version including at
  least +, -, *, /, \% (call fmod if using double values in C++),
  parentheses to alter precedence, =, +=, -=, *=, /=, \%=. (This is not
  a complete set of C/C++/Java operations, but to be complete, you'd
  have to allow data types or break some actual C++ rules. Data types
  are beyond the scope of this project!)
  now parse not just one but a sequence of ;-terminated expressions
  ending ultimately at an EOF marker; the sequence should be able to
  come from either a file or the keyboard; at the end of the sequence,
  display the values of all variables assigned to during the expressions.
  }
\subsubsection{Compiler Environment}
\srcfile{environment}
\subsubsection{Source}
\srcfile{../project/arithmetic-expression/Makefile}
\srccode{../project/arithmetic-expression/main.cpp}
\srccode{../project/arithmetic-expression/btree.hpp}
\srccode{../project/arithmetic-expression/exprset.hpp}
\srccode{../project/arithmetic-expression/exprtree.hpp}
\srccode{../project/arithmetic-expression/varstore.cpp}
\srccode{../project/arithmetic-expression/exprset.cpp}
\srccode{../project/arithmetic-expression/exprtree.cpp}
\subsubsection{Compiler Output}
\srcfile{../project/arithmetic-expression/compilerout}
\subsubsection{Program Output}
\srcfile{../project/arithmetic-expression/progout}