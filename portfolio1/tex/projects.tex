\section{Projects}
\subsection{Encryption Cipher}
\info{Encryption Cipher}{5}{
  Write a program that can perform encryption and decryption
  using an arbitrary substitution cipher. In this case, the
  encryption array is a random shuffling of the letters in the
  alphabet. Your program should generate a random encryption
  array, its corresponding decryption array, and use these to
  encode and decode a message. Allow for the saving and loading
  of encrypted messages by storing the 26 letter encryption key
  amongst the n encoded characters of the message. Note that for
  an $n$ character message, there will be $n+1$ slots amongst them.
  For a 1-character message, for instance, there is a slot before
  the character and a slot after the character. For a 2-character
  message, there are slots before and after the first character
  and after the second character. And so on... Make sure the extra
  \(26 \bmod (n+1)\) letters from the encryption key are located carefully
  to make the spread nice and even. (Note that when the message is
  longer than 25 characters, each letter of the key is alone by
  itself and, in fact, you are spreading the characters of the
  message amongst the 27 slots around the key values now.)}
\subsubsection{Compiler Environment}
\srcfile{environment}
\subsubsection{Source}
\srcfile{../project/cipher/Makefile}
\srccode{../project/cipher/main.cpp}
\subsubsection{Compiler Output}
\srcfile{../project/cipher/compilerout}
\subsubsection{Program Output}
\srcfile{../project/cipher/progout}